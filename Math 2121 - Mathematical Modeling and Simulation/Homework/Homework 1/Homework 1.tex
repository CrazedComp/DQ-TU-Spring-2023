% --------------------------------------------------------------
% This is all preamble stuff that you don't have to worry about.
% Head down to where it says "Start here"
% --------------------------------------------------------------
 
\documentclass[12pt]{article}
 
\usepackage[margin=1in]{geometry} 
\usepackage{amsmath,amsthm,amssymb}
 
\newcommand{\N}{\mathbb{N}}
\newcommand{\Z}{\mathbb{Z}}
 
\newenvironment{theorem}[2][Theorem]{\begin{trivlist}
\item[\hskip \labelsep {\bfseries #1}\hskip \labelsep {\bfseries #2.}]}{\end{trivlist}}
\newenvironment{lemma}[2][Lemma]{\begin{trivlist}
\item[\hskip \labelsep {\bfseries #1}\hskip \labelsep {\bfseries #2.}]}{\end{trivlist}}
\newenvironment{exercise}[2][Exercise]{\begin{trivlist}
\item[\hskip \labelsep {\bfseries #1}\hskip \labelsep {\bfseries #2.}]}{\end{trivlist}}
\newenvironment{reflection}[2][Reflection]{\begin{trivlist}
\item[\hskip \labelsep {\bfseries #1}\hskip \labelsep {\bfseries #2.}]}{\end{trivlist}}
\newenvironment{proposition}[2][Proposition]{\begin{trivlist}
\item[\hskip \labelsep {\bfseries #1}\hskip \labelsep {\bfseries #2.}]}{\end{trivlist}}
\newenvironment{corollary}[2][Corollary]{\begin{trivlist}
\item[\hskip \labelsep {\bfseries #1}\hskip \labelsep {\bfseries #2.}]}{\end{trivlist}}
 
\begin{document}
 
% --------------------------------------------------------------
%                         Start here
% --------------------------------------------------------------
 
%\renewcommand{\qedsymbol}{\filledbox}
 
\title{Homework 1}%replace X with the appropriate number
\author{Dean Quach\\ %replace with your name
MATH 2121 - Mathematical Modeling and Simulation} %if necessary, replace with your course title
\date{Janurary 24, 2023}
\maketitle

\hrule
\vspace{20pt}




\begin{exercise}{(a)} Explain why a "wave" arises that travels to the right.
\end{exercise}
$Sol^n:$\\

A wave starts among the $n=40$ agents because of the "initialization" step. \\
Notice for $n=40,$ \\
$\dfrac{n}{3}\approx 13.33 \implies round(13.33)=13.$\\

Then we do a logical operator, and in this case...\\
$n==round(1/3)\implies40=13?\implies F ~or~0$.\\
and then a double\\

Then for $j=1:80$ steps (we add pause so we can press enter 80 times),\\
%```
%fprintf('%d',x)
%```
%prints out integer digits (%d) of x (1 or 0).
%```
%fprintf('\n')
%```
%is just a new line. 
%
%Now 
%```
%x = ~x&x([end 1:end-1]);
%```
%makes a wave that travels to the right since, 

%- it redefines x to be the result of another logical operator, (&, which means evaluate both things)
%	- x=~x implies: any 1 is now a 0, and conversely any 0 is now a one. This also means that the initialization step changes and makes it so that the next entry to the right is "flagged", and on the next loop, it will make the $n=14$th position a 1.
%	- x=x([end $~$ 1:end-1]) implies: x redefined to be, take the end of the last vector, and put it to the front. Then take almost all entries 1:end-1 and put  it second. Basically moving everything over to the right by one, and putting the last entry to the beginning.






%\begin{theorem}{2.15} %You can use theorem, exercise, problem, or question here.  Modify x.yz to be whatever number you are proving
%An odd integer times an odd integer results in an odd integer.
%\end{theorem}
% 
%\begin{proof} %You can also use solution in place of proof.
%Assume m and n are both odd integers.\\
%Let m = 2k+1, and n = 2j+1\\
%So mn = 4kj+2k+2j+1\\
%Which factors into 2(2kj+k+j)+1.\\
%
%
%%Note 1: The * tells LaTeX not to number the lines.  If you remove the *, be sure to remove it below, too.
%%Note 2: Inside the align environment, you do not want to use $-signs.  The reason for this is that this is already a math environment. This is why we have to include \text{} around any text inside the align environment.
%By definition 2.9 (2kj+k+j) is an integer.\\
%so by definition 2.10 mn is an odd integer.\\\\
%\end{proof}
%
%\begin{theorem}{2.12} %You can use theorem, exercise, problem, or question here.  Modify x.yz to be whatever number you are proving
%The product of an odd integer and an even integer is odd.\\\\
%\end{theorem}
%\begin{proof} %You can also use solution in place of proof.
%The product of 2 times 3 is 6, which is an even number.
%
%
%\end{proof}








% --------------------------------------------------------------
%     You don't have to mess with anything below this line.
% --------------------------------------------------------------
 
\end{document}
