% --------------------------------------------------------------
% This is all preamble stuff that you don't have to worry about.
% Head down to where it says "Start here"
% --------------------------------------------------------------
 
\documentclass[12pt]{article}
 
\usepackage[margin=1in]{geometry} 
\usepackage{amsmath,amsthm,amssymb}
 
\newcommand{\N}{\mathbb{N}}
\newcommand{\Z}{\mathbb{Z}}
 
\newenvironment{theorem}[2][Theorem]{\begin{trivlist}
\item[\hskip \labelsep {\bfseries #1}\hskip \labelsep {\bfseries #2.}]}{\end{trivlist}}
\newenvironment{lemma}[2][Lemma]{\begin{trivlist}
\item[\hskip \labelsep {\bfseries #1}\hskip \labelsep {\bfseries #2.}]}{\end{trivlist}}
\newenvironment{exercise}[2][Exercise]{\begin{trivlist}
\item[\hskip \labelsep {\bfseries #1}\hskip \labelsep {\bfseries #2.}]}{\end{trivlist}}
\newenvironment{reflection}[2][Reflection]{\begin{trivlist}
\item[\hskip \labelsep {\bfseries #1}\hskip \labelsep {\bfseries #2.}]}{\end{trivlist}}
\newenvironment{proposition}[2][Proposition]{\begin{trivlist}
\item[\hskip \labelsep {\bfseries #1}\hskip \labelsep {\bfseries #2.}]}{\end{trivlist}}
\newenvironment{corollary}[2][Corollary]{\begin{trivlist}
\item[\hskip \labelsep {\bfseries #1}\hskip \labelsep {\bfseries #2.}]}{\end{trivlist}}
 
\begin{document}
 
% --------------------------------------------------------------
%                         Start here
% --------------------------------------------------------------
 
%\renewcommand{\qedsymbol}{\filledbox}
 
\title{Homework 1-24-23}%replace X with the appropriate number
\author{Dean Quach\\ %replace with your name
MATH 2121 - Mathematical Modeling and Simulation} %if necessary, replace with your course title
\date{Janurary 24, 2023}
\maketitle

\hrule
\vspace{20pt}



\begin{exercise}{Analysis of (a)}
\end{exercise}

It seems that each Monte Carlo simulation is quite different. Roughly half of the time, it is bimodal, i.e. two main spikes, and the 
other half of the time it is uniform or unimodal. In all cases, they do not follow any particular distribution, 
for example the bimodal runs, some seem symmetric, and some seem to have a long tail end. Similarly the unimodal. 
Also for the uniform, it generally is more "spikey" with a common max amount of runs ending with a position.

All cases are also not centered around 0. Most are below 0 with a lower bound of near $-1500$ to $-1600$, which 
implies the symmetry that occurs that I described before mostly happens around $-800$ (if it does happen). 
This is due to the middle agent which in general keeps the lower agent from walking past 5. Also if the middle one goes up, it may 
not go up as much since it is in the middle and has an upper walker which keeps it frrom increasing the range of the bottom walker 
(in the positive/up direction). So there is "more inscentive" for the bottom walker to keep decreasing. Also if the middle walker goes 
down/negative, then the bottom walker must decrease as well.

Also quick note, since the middle walker affects the lower walker, some times the range can be 
$-100$ to $400$. Which in this case, implies that the middle walker stayed in the middle (and didn't, on average, choose to 
increase/decrease too much).

Code source: temple\_abm\_random\_walk\_2d\_mod\\
from class.

\begin{exercise}{Analysis of (b)}
\end{exercise}
I do not know if this is meant to happen, but when looking at a single walk (sample), since the middle walker starts at the midpoint 
between the top and bottom walker, then the next step has a $50\%$ chance to go up or down. But it always goes the same 
direction and stays in the middle. I don't know if this is because its basing off of the initial condition or...

As for the histogram, I don;t know what I did wrong, but I get a uniform distribution. This is because of my poor code from the 
previous block. Every walk ends in a specific final position, and the number is usually 100 for each. But the final position generally 
ranges from $0\to 2$, $-2\to 0$, or $0\to 16$. 


\begin{exercise}{Analysis of (c)}
\end{exercise}

I am stuck on update rule and why it isn't working.





%\begin{theorem}{2.15} %You can use theorem, exercise, problem, or question here.  Modify x.yz to be whatever number you are proving
%An odd integer times an odd integer results in an odd integer.
%\end{theorem}
% 
%\begin{proof} %You can also use solution in place of proof.
%Assume m and n are both odd integers.\\
%Let m = 2k+1, and n = 2j+1\\
%So mn = 4kj+2k+2j+1\\
%Which factors into 2(2kj+k+j)+1.\\
%
%
%%Note 1: The * tells LaTeX not to number the lines.  If you remove the *, be sure to remove it below, too.
%%Note 2: Inside the align environment, you do not want to use $-signs.  The reason for this is that this is already a math environment. This is why we have to include \text{} around any text inside the align environment.
%By definition 2.9 (2kj+k+j) is an integer.\\
%so by definition 2.10 mn is an odd integer.\\\\
%\end{proof}
%
%\begin{theorem}{2.12} %You can use theorem, exercise, problem, or question here.  Modify x.yz to be whatever number you are proving
%The product of an odd integer and an even integer is odd.\\\\
%\end{theorem}
%\begin{proof} %You can also use solution in place of proof.
%The product of 2 times 3 is 6, which is an even number.
%
%
%\end{proof}








% --------------------------------------------------------------
%     You don't have to mess with anything below this line.
% --------------------------------------------------------------
 
\end{document}
