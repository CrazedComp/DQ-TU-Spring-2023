\documentclass[12pt]{book}
\usepackage[width=4.375in, height=7.0in, top=1.0in, papersize={5.5in,8.5in}]{geometry}
\usepackage[pdftex]{graphicx}
\usepackage{amsmath}
\usepackage{amssymb}
\usepackage{tipa}
%\usepackage{txfonts}
\usepackage{textcomp}
%\usepackage{array}
%\usepackage{xy}
\usepackage{fancyhdr}

\pagestyle{fancy}
\renewcommand{\chaptermark}[1]{\markboth{#1}{}}
\renewcommand{\sectionmark}[1]{\markright{\thesection\ #1}}
\fancyhf{}
\fancyhead[LE,RO]{\bfseries\thepage}
\fancyhead[LO]{\bfseries\rightmark}
\fancyhead[RE]{\bfseries\leftmark}
\renewcommand{\headrulewidth}{0.5pt}
\renewcommand{\footrulewidth}{0pt}
\addtolength{\headheight}{0.5pt}
\setlength{\footskip}{0in}
\renewcommand{\footruleskip}{0pt}
\fancypagestyle{plain}{%
\fancyhead{}
\renewcommand{\headrulewidth}{0pt}
}
%
%\parindent 0in
\parskip 0.05in

\usepackage{hyperref}
\hypersetup{
	colorlinks=true,
	linkcolor=blue,
	filecolor=magenta
	urlcolor=cyan
	urlstyle{same}
	}
%\href{www.aaa.com}{This Link}


\usepackage{amsthm}
\newenvironment{theorem}[2][Theorem]{\begin{trivlist}
\item[\hskip \labelsep {\bfseries #1}\hskip \labelsep {\bfseries #2.}]}{\end{trivlist}}
\newenvironment{lemma}[2][Lemma]{\begin{trivlist}
\item[\hskip \labelsep {\bfseries #1}\hskip \labelsep {\bfseries #2.}]}{\end{trivlist}}
\newenvironment{exercise}[2][Exercise]{\begin{trivlist}
\item[\hskip \labelsep {\bfseries #1}\hskip \labelsep {\bfseries #2.}]}{\end{trivlist}}
\newenvironment{example}[2][Example]{\begin{trivlist}
\item[\hskip \labelsep {\bfseries #1}\hskip \labelsep {\bfseries #2.}]}{\end{trivlist}}
\newenvironment{Recall}[2][Recall]{\begin{trivlist}
\item[\hskip \labelsep {\bfseries #1}\hskip \labelsep {\bfseries #2.}]}{\end{trivlist}}
\newenvironment{remark}[2][Remark]{\begin{trivlist}
\item[\hskip \labelsep {\bfseries #1}\hskip \labelsep {\bfseries #2.}]}{\end{trivlist}}
\newenvironment{proposition}[2][Proposition]{\begin{trivlist}
\item[\hskip \labelsep {\bfseries #1}\hskip \labelsep {\bfseries #2.}]}{\end{trivlist}}
\newenvironment{corollary}[2][Corollary]{\begin{trivlist}
\item[\hskip \labelsep {\bfseries #1}\hskip \labelsep {\bfseries #2.}]}{\end{trivlist}}
\newenvironment{definition}[2][Definition]{\begin{trivlist}
\item[\hskip \labelsep {\bfseries #1}\hskip \labelsep {\bfseries #2.}]}{\end{trivlist}}

\newtheorem*{theorem*}{Theorem}
\newtheorem*{corollary*}{Corollary}
\newtheorem*{lemma*}{Lemma}
\newtheorem*{definition*}{Definition}
\newtheorem*{remark*}{Remark}
\newtheorem*{example*}{Example}


\usepackage{fancyvrb} 
%\begin{Verbatim}[numbers=left, frame=single, formatcom=\color{red}]

\begin{document}
\frontmatter
%
\chapter*{\Huge \center Mathematical Modeling and Simulation Notes}
\thispagestyle{empty}
%{\hspace{0.25in} \includegraphics{./ru_sun.jpg} }
\section*{\huge \center Dean Quach}



\newpage
\section*{\center \normalsize Course Info}
\subsection*{\center \normalsize Mathematical Modeling and Simulation}

\begin{center}
Janurary 17, 2023 to May 9, 2023\\
TR --- 10:00 am - 12:20 pm --- Wachman 10\\
Prof. Mrs. Rujeko Chinomona\\
CRN 48064

3 credits $\implies{}$ 6 hours of homework
\end{center}

\subsection*{\center \normalsize Text}
\begin{center}
No course textbook, however all readings are on the \href{https://templeu.instructure.com/courses/124350}{Canvas page} under \href{https://templeu.instructure.com/courses/124350/pages/topic-schedule}{"Topic Schedule"}.
\end{center}


\newpage
\section*{\center \normalsize Other Notes}
\begin{center}
Homework due $\approx$ weekly on friday, through Github. Generally, MATLAB excercises.

No exams, but eventually, everyother week there will be a check-in for the project.

Final project presentation 1 week before end of semester. 

Project files/report due 1 week after semester ends.\\
\end{center}
\vspace{50pt}
\begin{center}
Office Hours: Wachman 512

TR 1:30-3:30pm
\end{center}


%
\chapter*{\center \normalsize Spring 2023\\
Math 2121\\
Section 001\\}
%
















%
\tableofcontents
%
\mainmatter
%













\chapter{Git/Github and Intro to Agent Based Modeling}
For the Git/Github Setup, its simple enough, go download:
\begin{enumerate} 
	\item Make an account \href{https://github.com/}{here}
	\item \href{https://notepad-plus-plus.org/downloads/}{Notepad++}
	\item \href{https://git-scm.com/downloads}{Git}
	\item \href{https://desktop.github.com/}{Github Desktop}
\end{enumerate}
in that order. The main part being ``Notepad++" before ``Git". This is so that when you configure the installation of ``Git", you can choose ``Notepad++" as the default terminal, instead of ``Vim" editor (I believe). Also we will be using MATLAB in this course.

\newpage 
\section{Types of Statistics}
\begin{itemize}
	\item Deterministic
	\item Stochastic
	\begin{itemize}
		\item which is Randomness
	\end{itemize}
\end{itemize}
both of which you will see in various models.

\vspace{50pt}
The focus of this class will be
\section{Agent Based Modeling (or ABM)}
Specifically, Complex adaptive systems

We'll look at individual agents, which have
\begin{itemize}
	\item Emergent behavior
	\item macro scale behavior $>$ micro scale behavior\\
\end{itemize}


\begin{definition*}
Computational study of processes modeled as\\ dynamic systems of interacting agents
\end{definition*}
-   Key idea: identifying individuals/agents


\newpage
\noindent Things to look for can include
\begin{itemize}
	\item Agents: organisms, humans, businesses, institutions
	\item Unique: different size location, resource reserves, history, etc.
	\item Interacting locally: agents usually interact with neighbors and not all agents in geographic space or network.
	\item Autonomous:
	\item Adaptive:
\end{itemize}

\noindent Key things to find (what are ABMs composed of)
\begin{itemize} 
	\item Number of agents
	\item Set of decision making for each agentSet of learning rules for each agent
	\item A space in which the agents can move/operate and an environment in which they can interact
	\begin{itemize} 
		\item(It's kind of like game design)
	\end{itemize}
\end{itemize}



\section{ABM Simulation}
\begin{enumerate}
	\item Setup environment and agents
	\item Run for each time step
	\item (Do Something)
\end{enumerate}

\noindent \textbf{E.g.}
\begin{itemize}
	\item Agents: people at temple
	\item Characteristic: 

	\begin{itemize}
		\item masks or not
		\item age/height/weight/etc.
	\end{itemize}
	\item Rules: (what to keep track of)

	\begin{itemize}
		\item Simplify to within one room
		\item Track mask or not
	\end{itemize}

	\item Movement (gas model for example)
	\item Probability of collisions, getting infected, etc.
	\begin{itemize}
		\item Think stochastic video games, \\movie simulation/cgi/large wars and ragdolls
	\end{itemize}
\end{itemize}



\section{ABM Properties and Methodology}
\noindent The general properties of an ABM deals with
\begin{enumerate}
	\item Individuals/Agents
	\item Act according to self interest
	\item Can interact with each other
	\item Can interact with the environment
\end{enumerate}

We'll look at economics, social studies, biology, etc. as a broad overview.\\

\newpage
\noindent For example, for Epidemiology
\begin{itemize}
\item Disease modeling (and spreading through out a population
	\item e.g. Covid-19, HIV, SARS, Flu, Measles, Chickenpox
\end{itemize}

\noindent So for modeling, what would you need to consider? \\(2 main ways)



\subsection{Way 1: For an ABM}
Represent individuals in population (which may be humans, different animals... or be spiecific to human like old/young/ healthy/etc.)
Simulate what happens when individuals meet.\\


\noindent \textbf{E.g. - Characteristics of disease}
\begin{itemize}
	\item How does the disease spread? 
		\begin{itemize}
		\item High/Low Probability of infecting individuals (old/young or healthy/unhealthy)
		\end{itemize}
	\item Time
		\begin{itemize}
		\item recovery speed
		\item period of infection
		\item period of exposure before you get infected
		\end{itemize}
\end{itemize}



\noindent Attributes of agents
\begin{itemize}
\item Age
\item Gender
\item Immunocomprimised?
\item Location
	\begin{itemize}
	\item Related: Healthy vs. Unhealthy
	\end{itemize}
\end{itemize}

\noindent Now that we have recognized/observed these properties, we need to
\begin{itemize}
\item Define an environment
\item Define a set of rules, by which agents move/interact. The interaction can be 
	\begin{itemize}
	\item agent - agent
	\item agent - environment
	\end{itemize}
\item Create Code
\item Run Simulation
	\begin{itemize}
	\item Discrete steps (a.k.a. time steps)
	\end{itemize}
\item Analyze Results
\end{itemize}



\noindent Side note: ABM's can get large too quickly with \\
$O(f(x)) \rightarrow \infty$. So before you start your modeling task, you need to identify the key questions your model needs to answer.



\subsection{Way 2: Quickly (we won't talk about this for now)}
\noindent This method by groups "loses resolution",
\begin{itemize}
\item Group your population (e.g. SIR model)
	\begin{itemize}
	\item Suseptible $=s$
	\item Infected $=i$
	\item recovered $=r$
	\end{itemize}
\end{itemize}

And this leads to a differential equation model, for an SIR model with variables...
you'll analyize/model with 
$$\frac{\partial S}{\partial t},~\frac{\partial i}{\partial t},~\text{and}~\frac{\partial r}{\partial t}.$$



\section{About Github}

Clone Repository - a.k.a. "download"\\
Make a MATLAB folder on desktop/USB drive for organizing files. 
e.g. MATH2121Sp23
\begin{itemize}
\item MATH2121
	\begin{itemize}
	\item HW1
	\item HW2
	\item ...
	\end{itemize}
\end{itemize}

Refer to the ``github" cheat sheet and ``gittools" pdf for things you need to know. Namely,
\begin{itemize}
\item Clone
\item Commit
\item Fetch
\item Push
\item Pull
\end{itemize}













\chapter{Elaborating on \\Stochasticity/Randomness}
\indent Recall we're looking at "Mathematical Modeling"
and to be honest, the following equation shall be implemented as much to heart as you possibly can.

$$Model = Simplification~of~reality$$

\noindent Part of this includes choosing the correct type of statistical measurement method.

\section{Deterministic vs. Stochastic} 
Preface: It is not specifically one or the other, there can be a little bit of both in eachother.\\


\newpage
\subsection{``Deterministic"}
...generally predicts what will happen.
\begin{itemize}
\item $F=ma$
\item Physical laws modeled by differential equations
\item Initial conditions leads to same output everytime.
\end{itemize}

\subsection{``Stochasticity"}
...are generally random events where different results can be obtained from the same initial condition.

--- Why introduce stochasticity in a model?

\noindent Usually because of some human/living being indecisiveness := Natural variability. 

\noindent \textbf{E.g.}
\begin{itemize}
\item modeling crowd behavior
\item model consumer behavior
\item model behavior on stock market
\end{itemize}

\noindent Use stochasticity to assign initial conditions,\\
and you can Model variable outcome processes.\\

\noindent Random numbers in models, where\\
you pick random \# distribution.

The main point is that Stochasticity/Randomness has
\begin{itemize} 
\item Natural Variability
\item Stochastic Variable to model complex processes
\item Initialize model
\item Monte Carlo Simulation (Run it many times)
	\begin{itemize} 
	\item Use a histogram to visualize different frequences
	\end{itemize} 
\end{itemize} 



\section{Continuous vs. Discrete \\Distributions}
\indent For a random variable, which one to use and when?

\noindent \textbf{For continuous distributions}
\begin{itemize}
\item normal distribution
\item uniform distribution
\item log-normal distribution
\end{itemize}






\chapter{Random Walks}
There are many takes/definitions on what a random walk is.
\begin{itemize}
\item Process by which randomly moving objects wander away from their starting points.
\item Stochastic Process for determining the probable location of a point subject to random motions, givin the probabilities of moving some distance in some direction.
\end{itemize}

\section{1D Random Walks}
You can imagine a number line, where 
\begin{itemize}
\item you start at 0
\item you move $\pm 1$ steps (right or left/up and down), where we can set a probability measure
	\begin{itemize}
	\item assume we set $P(\pm 1)=.5$
	\end{itemize}
\item Per discrete step in time, we do an action. 
\item One of the main questions: Where would you end up (probability)
\end{itemize}



\section{2D Random Walks}

\hspace{\parindent}This would similarly have two dimensions, which implies a whole plane of freedom.

\noindent Random walks can represent: 
\begin{enumerate}
\item Path of molecules in a gas/molecules in a \\gas/liquid (Brownian Motion)
\item Path of an animal looking for food
\item Short term flucuating Stock
\item Diffusion 
\end{enumerate}

\noindent Questions:
\begin{enumerate}
\item What is the expected position of the particle after $n$ steps?
\item How far does the particle travel after $n$ steps?
\end{enumerate}










\chapter{Statistical Measurements}
\begin{definition*}Mean
$$\bar{x}=\frac{1}{n} \sum\limits_{i=1}^{n} x_{i}=\frac{x_{1}+x_{2}+\cdots+x_{n}}{n}$$
\end{definition*}

\begin{definition*}Variance\footnote[1]{(Note: Matlab will default to $\frac{1}{n-1}$ (basel's correction) i.e. the entire population)}\\
\end{definition*}
\indent If looking at the entire population
$$var(x)=\frac{1}{n-1} \sum\limits_{i=1}^{n} (x_{i}-\bar{x})^{2}$$


If looking at a sample of popultaion
$$var(x)=\frac{1}{n} \sum\limits_{i=1}^{n} (x_{i}-\bar{x})^{2}$$



\begin{definition*}Standard Deviation
$$\sigma = \sqrt{var(x)}=\sqrt{\frac{1}{n-1} \sum\limits_{i=1}^{n} (x_{i}-\bar{x})^{2}}$$
\end{definition*}







\chapter{Simulated Annealing}

...metallurgy...

\noindent Instead of picking best move, pick a random move. 

Set probability of accempting a bad move, $P$ will go down as you take more steps. \\

\noindent If the temp is
\begin{itemize}
\item High $\implies{}$ more likely to accept bad move
\item Low $\implies{}$ more likely to reject a bad move.
\end{itemize}


\newpage
\section{Pseudocode}

Let $x=x_{0}$, and $f(x)=f(x_{0})$.

Let $n$ be total $\#$ of steps in random walk.


\begin{Verbatim}[numbers=left, frame=single, formatcom=\color{black}]
for j=1:n
then, Set temperature

Pick a random move, 
x_{mean}=x+random variable
\end{Verbatim}





\backmatter
%
\begin{thebibliography}{99}
\end{thebibliography}
\end{document}
