\documentclass[12pt]{report}

\usepackage{fullpage}
\renewcommand{\baselinestretch}{1}

\usepackage{amsmath, amssymb}
\usepackage[utf8]{inputenc}

\usepackage[table]{xcolor}

\usepackage{anyfontsize}
\usepackage{mathptmx}
\usepackage{t1enc}
\DeclareMathAlphabet{\mathcal}{OMS}{cmsy}{m}{n}
\usepackage{mathrsfs}

\usepackage{tikz}

\usepackage{float}

\usepackage{cancel}

\begin{document}


\begin{center}
\large{Spring 2023}
\end{center}

\vspace{20pt}

\begin{center}
\huge{\textbf{Practice Midterm \#1 Exam}}
\end{center}

\begin{center}
\Large{MATH 3031}
\end{center}

\begin{center}
\large{\textbf{Department of Mathematics \\ Temple University}}
\end{center}

\vspace{30pt}

\begin{center}
\large{Februrary 22, 2023}
\end{center}

\vspace{10pt}

\begin{center}
\large{\textbf{Name:} \hrulefill}
\end{center}

\begin{center}
\large{\textbf{Instructor/Section:} \hrulefill}
\end{center}




\vspace{30pt}


\begin{minipage}[b]{80mm}
\begin{tabular}{c}
\noindent This exam consists of 6 questions. \\

\noindent This exam will take \\1hr + 10min to complete. \\

\noindent A 4-function calculator may be used.\\

\noindent Show all relevant work.\\
\noindent \textbf{No work, no credit.}\\

\noindent \textbf{Good Luck!}\\
\end{tabular}
\end{minipage}
\begin{minipage}[b]{80mm}
\begin{tabular}{ |c|c|c| }
	\hline
	\textbf{Question} & \textbf{Max} & \textbf{Points} \\
	\hline
	\hline
	1& \textbf{8} & \\
	\hline
	2& \textbf{4} & \\
	\hline
	3& \textbf{8} & \\
	\hline
	4& \textbf{4} & \\
	\hline
	5& \textbf{4} & \\
	\hline
	6& \textbf{8} & \\
	\hline
	\hline
	\textbf{Total} & \textbf{36} & \\
	\hline
\end{tabular}
\end{minipage}

\pagebreak








\noindent \fbox{8pts} $~$\textbf{1.} We have an urn with 3 red and 10 blue balls. We draw 5 balls, one by one, without replacement. 
\begin{itemize}
\item [(a)] State $\Omega$. Find the probability that the colors we see in order are blue, blue, blue, red, blue.		%normal ordered, n choose k, but for each color
\item [(b)] Find the probability that our sample of balls were all blue.
\item [(c)] Consider the new sample space $\Omega'$, which is constructed after part (b), i.e. remove 5 blue balls. Suppose we changed our sample size to $k=4$. State $\Omega'$. Find the probability of the event $A=$ \{2 are red and 2 are blue\}.
\item [(d)]  Find the probability that the new sample $A$ has at least 2 blue balls. ($\omega'_i\in\Omega'$)
\end{itemize}
%\textit{Solution. } \\
%\begin{align*}
%P(b,b,b,r,b)&=\frac{\cancel{10}}{13}\cdot \frac{\cancel{9}}{\cancelto{3}{12}}\cdot\frac{\cancelto{2}{8}}{11}\cdot\frac{\boxed{3}}{\cancel{10}}\cdot\frac{7}{\cancel{9}}\tag{since this one's red}\\
%&=\frac{2\cdot\cancel{3}\cdot7}{13\cdot\cancel{3}\cdot11}\\
%&=\frac{2\cdot7}{13\cdot11}=\frac{14}{143}\approx 0.097902.~~~\square
%\end{align*}
%
%\begin{itemize}
%\item [(b)] Find the probability that our sample of balls were all blue.					%normal unordered, n choose k, but #P(b)/#totalP
%\end{itemize}
%\textit{Solution. } \\
%\begin{align*}
%P(b,b,b,b,b)&=\frac{\displaystyle{10\choose5}}{\displaystyle{13\choose5}}=\dfrac{\dfrac{10\cdot9\cdot8\cdot7\cdot6}{5\cdot4\cdot3\cdot2\cdot1}}{\dfrac{13\cdot12\cdot11\cdot10\cdot9}{5\cdot4\cdot3\cdot2\cdot1}}\\
%&=\dfrac{10\cdot9\cdot8\cdot7\cdot6}{\cancel{5\cdot4\cdot3\cdot2\cdot1}}\left(\dfrac{\cancel{5\cdot4\cdot3\cdot2\cdot1}}{13\cdot12\cdot11\cdot10\cdot9}\right)\\
%&=\dfrac{\cancel{10\cdot9}\cdot8\cdot7\cdot6}{13\cdot12\cdot11\cdot\cancel{10\cdot9}}=\dfrac{\cancelto{2}{8}\cdot7\cdot6}{13\cdot\cancelto{3}{4}\cdot11}\\
%&=\dfrac{2\cdot7\cdot\cancelto{2}{6}}{13\cdot\cancelto{1}{3}\cdot11}=2\left(\boxed{\dfrac{14}{143}}\right) \approx 2(0.097902)=0.195804.~~~\square \tag{from part (a)}
%\end{align*}
%
%\begin{itemize}
%\item [(c)] Consider the new sample space $\Omega'$, which is constructed after part (b), i.e. remove 5 blue balls. Suppose we changed our sample size to $k=4$. State $\Omega'$. Find the probability of the event $A=$ \{2 are red and 2 are blue\}.   %normal unordered, n choose k, but #P(b+1)#P(r)/#totalP
%\end{itemize}
%\textit{Solution. } \\
%Let $\Omega'=\Omega\backslash\{b,b,b,b,b\}=\{10b,3r\}\backslash\{5b\}=\{\text{3 red and 5 blue}\}$.
%Now,
%\begin{align*}
%P(\{2r,2b\})=P(A)&=\dfrac{\displaystyle{3\choose2}{5\choose2}}{\displaystyle{8\choose4}}=\dfrac{\dfrac{3\cdot2}{2\cdot1}\left(\dfrac{5\cdot4}{2\cdot1}\right)}{\dfrac{8\cdot7\cdot6\cdot5}{4\cdot3\cdot2\cdot1}}\\
%&=\dfrac{3\cdot2\cdot5\cdot\cancel{4}}{\cancel{4}}\left( \dfrac{\cancel{4}\cdot\cancelto{1}{3}\cdot\cancel{2}\cdot1}{\cancel{8}\cdot7\cdot\cancelto{2}{6}\cdot5} \right)\\
%&=\dfrac{3\cdot\cancel{2\cdot5}}{7\cdot\cancel{2\cdot5}}=\dfrac{3}{7}\approx0.4285714.~~~\square
%\end{align*}
%
%
%\begin{itemize}
%\item [(d)]  Find the probability that the new sample $A$ has at least 2 blue balls. ($\omega'_i\in\Omega'$) 		%sum of probabilities, dependent (without replacement), P(2g)+P(3g)P+P(4g)+P(4g)
%\end{itemize}
%\textit{Solution. } \\
%Let $P(A)=P(kb)$ such that $k\in\mathbb{N}$.
%Now, since at least 2 blue (and cannot exceed 4 since $\#\omega=4$),\\
%then $\implies{}\displaystyle\sum\limits_{k=2}^{4}P(kb)=P(2b)+P(3b)+P(4b)$
%\begin{align*}
%&=\dfrac{\displaystyle{5\choose2}{3\choose2}}{\displaystyle{8\choose3}}+\dfrac{\displaystyle{5\choose3}{3\choose1}}{\displaystyle{8\choose3}}+\dfrac{\displaystyle{5\choose4}\cancel{{3\choose0}}}{\displaystyle{8\choose3}}\\\\
%&=\dfrac{  \dfrac{5\cdot4}{2\cdot1}\left(\dfrac{3\cdot\cancel{2}}{\cancel{2}\cdot1}\right)  +  \dfrac{5\cdot4\cdot3}{3\cdot2\cdot1}(3)   +   \dfrac{5\cdot4\cdot3\cdot2}{4\cdot3\cdot2\cdot1}}        {\dfrac{8\cdot7\cdot6\cdot5}{4\cdot3\cdot2\cdot1}}\\\\
%&=  \left[   \dfrac{5\cdot4\cdot3}{2}+\dfrac{5\cdot4\cdot3\cdot3}{3\cdot2} + \dfrac{5\cdot\cancel{4\cdot3\cdot2}}{\cancel{4\cdot3\cdot2}\cdot1}   \right]   \left(\dfrac{4\cdot3\cdot2\cdot1}{8\cdot7\cdot6\cdot5}\right)\\\\
%&= \dfrac{5\cdot2\cdot3+5\cdot2\cdot3+5}{2\cdot7\cdot5}=\dfrac{65}{70}=\dfrac{13}{14}\approx0.9285714.~~~\square
%\end{align*}


\pagebreak






% Note, there needs to be an enter/space between eqach question, as well as the vspace choice
\noindent \fbox{4pts} $~$\textbf{2.} If $P(A)=\dfrac{1}{2},~P(B)=\dfrac{1}{3},~\text{and}~P(A\cap B)=\dfrac{1}{4}$,
\begin{itemize}
\item [(a)] Find $P(A^C\cap B^C)$		%demorgan complement to union, then complement, then inclusion exclusion 
\item [(b)] Find $P(A \cup B^C)$
\end{itemize}
%\textit{Solution. } \\
%Notice $P(A^C\cap B^C)=P((A\cup B)^C)$ by a de Morgan Law.\\
%Now note using the complement definition, we have\\
%$P((A\cup B)^C)=1-P(A\cup B)$.\\
%And of course, by the inclusion exclusion principle, we get
%\begin{align*}
%1-P(A\cup B)&=1-[P(A)+P(B)-P(A\cap B)]\\
%&=1-\left(\frac{1}{2}+\frac{1}{3}-\frac{1}{4}\right)\\
%&=1-\left(\frac{12+8-6}{24}\right)\\
%&=1-\left(\frac{14}{24}\right)=\frac{12}{12}-\left(\frac{7}{12}\right)=\frac{5}{12}\approx0.41666.~~~\square
%\end{align*}

%\begin{itemize}
%\item [(b)] Find $P(A \cup B^C)$			%inclusion exclusion,complements, cancelation property of complement P(a)+P(a^C) (and rearranging it)
%\end{itemize}
%\textit{Solution. } \\
%By the inclusion exclusion principle, we have \\
%$P(A \cup B^C)=P(A)+P(B^C)-P(AB^C)$.\\
%
%\noindent Now we know by definition, $P(A)=P(AB)+P(AB^C)$ (which is a property of complements).\\
%But of course this implies $P(AB^C)=P(A)-P(AB)$.\\
%
%\noindent So now, 
%\begin{align*}
%P(A \cup B^C)&=P(A)+P(B^C)-P(AB^C)\\
%&=P(A)+P(B^C)-P(A)-P(AB)\\
%&=\cancel{P(A)}+(1-P(B))-\cancel{P(A)}-P(AB)	\tag{by $def^n$ of complement}\\
%&=1-P(B)-P(AB)\\
%&=\frac{12}{12}-\frac{4}{12}-\frac{3}{12}=\frac{5}{12}\approx0.41666.~~~\square
%\end{align*}
\pagebreak





\noindent \fbox{8pts} $~$\textbf{3.} An urn contains 3 balls labeled 2, 3, and 4. We draw 2 balls one by one at random with replacement. Let $X$ be the sum of the two numbers of the sample.
\begin{itemize}
\item [(a)] Find the possible values of $X$.		%pure observation
\item [(b)] Find the probability mass function of $X$.
\item [(c)] Let $Y=3X-21$. Find the probability mass function of $Y$.
\item [(d)] Prove that $X$ and $Y$ are dependent for some $(i,j)\in\Omega^2$.
\end{itemize}
%\textit{Solution. } $X=\{5,6,7\}.~~~\square$
%
%\begin{itemize}
%\item [(b)] Find the probability mass function of $X$.			%pure calculation
%\end{itemize}
%\textit{Solution. } 
%\end
%
%\begin{itemize}
%\item [(c)] Let $Y=3X-21$. Find the probability mass function of $Y$.			%just scale X, same prob
%\end{itemize}
%
%\begin{itemize}
%\item [(d)] Prove that $X$ and $Y$ are dependent for some $(i,j)\in\Omega^2$.
%\end{itemize}
\pagebreak







\noindent \fbox{4pts} $~$\textbf{4.} We have 3 urns. Urn I has 1 green and 2 red balls, urn II has 2 green and 3 red balls, and urn III has 3 green and 4 red balls. 
\begin{itemize}
\item [(a)] We first choose an urn at ransom and then choose a ball randomly from the chosen urn. Find the probability that the ball is red. %conditional prob, useing \sum multiplcation rule
\item [(b)] Suppose we draw a random ball from urn I and transfer it to urn III. Then we choose a ball randomly from urn III. What is the probability that we draw a green ball from urn III.
%multiplication rule over the intersection of probs. P(r & 1 cap r &3)=P(R I)*P(R III | R I) + P(R III \cap G I)
\end{itemize}
\pagebreak





\noindent \fbox{4pts} $~$\textbf{5.} There is a new test for a disease that occurs in about 0.01\% of the population. It detects the disease 99\% of the time. However it has a false positive rate of 10\%. 
\begin{itemize}
\item [(a)] What is the probability that the person's result is positive.		%\sum of conditional prob
\item [(b)] What is the probability that the person actually has the disease if they test positive.		%bayes, use part a 
\end{itemize}
\pagebreak





\noindent \fbox{8pts} $~$\textbf{6.} Suppose $A,B,C\in\mathcal{F},~s.t.~A,B,C$ are mutually independent. \\

Set $P(A)=.1,~P(B)=.2,~P(C)=.3$. 
\begin{itemize}
\item [(a)] $P(A\cap B \cap C^C)$ 		%independence rules
\item [(b)] $P(A\cup C)$			%inclusion/exclusion principle, then independence on P(AC)
\item [(c)] $P((A\cup C)\cap B)$
\item [(d)] Prove that $A\cup C$ is independent of B.
\end{itemize}
\pagebreak



















\end{document}






