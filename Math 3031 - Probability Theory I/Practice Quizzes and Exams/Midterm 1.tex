\documentclass[12pt]{report}

\usepackage{fullpage}
\renewcommand{\baselinestretch}{1}

\usepackage{amsmath, amssymb}
\usepackage[utf8]{inputenc}

\usepackage[table]{xcolor}

\usepackage{anyfontsize}
\usepackage{mathptmx}
\usepackage{t1enc}
\DeclareMathAlphabet{\mathcal}{OMS}{cmsy}{m}{n}
\usepackage{mathrsfs}

\usepackage{tikz}

\usepackage{float}

\begin{document}


\begin{center}
\large{Fall 2022}
\end{center}

\vspace{20pt}

\begin{center}
\huge{\textbf{Practice Midterm Exam}}
\end{center}

\begin{center}
\Large{MATH 3031}
\end{center}

\begin{center}
\large{\textbf{Department of Mathematics \\ Temple University}}
\end{center}

\vspace{30pt}

\begin{center}
\large{Februrary 22, 2023}
\end{center}

\vspace{10pt}

\begin{center}
\large{\textbf{Name:} \hrulefill}
\end{center}

\begin{center}
\large{\textbf{Instructor/Section:} \hrulefill}
\end{center}




\vspace{30pt}


\begin{minipage}[b]{80mm}
\begin{tabular}{c}
\noindent This exam consists of 6 questions. \\

\noindent This exam will take 50 minutes to complete. \\

\noindent NO calculators.\\

\noindent Show all relevant work.\\
\noindent \textbf{No work, no credit.}\\

\noindent \textbf{Good Luck!}\\
\end{tabular}
\end{minipage}
\begin{minipage}[b]{80mm}
\begin{tabular}{ |c|c|c| }
	\hline
	\textbf{Question} & \textbf{Max} & \textbf{Points} \\
	\hline
	\hline
	1& \textbf{14} & \\
	\hline
	2& \textbf{12} & \\
	\hline
	3& \textbf{15} & \\
	\hline
	4& \textbf{12} & \\
	\hline
	5& \textbf{12} & \\
	\hline
	6& \textbf{20} & \\
	\hline
	\hline
	\textbf{Total} & \textbf{20} & \\
	\hline
\end{tabular}
\end{minipage}

\pagebreak








\noindent \fbox{7pts} $~$\textbf{1.} We have an urn with 3 red and 10 blue balls. We draw 5 balls, one by one, without replacement. 
\begin{itemize}
\item [(a)] Find the probability that the colors we see in order are blue, blue, blue, red, blue.		%normal ordered, n choose k, but for each color
\item [(b)] Find the probability that our sample of balls were all blue.					%normal unordered, n choose k, but #P(b)/#totalP
	\begin{itemize}
	\item [(i)] Using part (b), find the probability that the next two balls are red and blue (after the first 5 were blue).   %normal unordered, n choose k, but #P(b+1)#P(r)/#totalP
	\end{itemize}
\item [(c)]  Find the probability that our sample of 5 balls has at least 2 green balls. 		%sum of probabilities, dependent (without replacement), P(2g)+P(3g)P+P(4g)+P(4g)
\end{itemize}
\pagebreak






% Note, there needs to be an enter/space between eqach question, as well as the vspace choice
\noindent \fbox{4pts} $~$\textbf{2.} If $P(A)=\dfrac{1}{2},~P(B)=\dfrac{1}{3},~\text{and}~P(A\cap B)=\dfrac{1}{4}$,
\begin{itemize}
\item [(a)] Find $P(A^C\cap B^C)$		%demorgan complement to union, then complement, then inclusion exclusion 
\item [(b)] Find $P(A \cup B^C)$			%inclusion exclusion,complements, cancelation property of complement P(a)+P(a^C) (and rearranging it)
\end{itemize}
\pagebreak





\noindent \fbox{8pts} $~$\textbf{3.} An urn contains 3 balls labeled 2, 3, and 4. We draw 2 balls one by one at random with replacement. Let $X$ be the sum of the two numbers on the sample.
\begin{itemize}
\item [(a)] Find the possible values of $X$.		%pure observation
\item [(b)] Find the probability mass function of $X$.			%pure calculation
\item [(c)] Let $Y=3X-21$. Find the probability mass function of $Y$			%just scale X, same prob
\item [(d)] Prove that $X$ and $Y$ are dependent for some $(i,j)\in\Omega^2$.			%pf by contradiction, assume they are independent. this is impossible since P(x=3,y=4)
														%=P({3,4})*P({})
\end{itemize}
\pagebreak







\noindent \fbox{4pts} $~$\textbf{4.} We have 3 urns. Urn I has 1 green and 2 red balls, urn II has 2 green and 3 red balls, and urn III has 3 green and 4 red balls. 
\begin{itemize}
\item [(a)] We first choose an urn at ransom and then choose a ball randomly from the chosen urn. Find the probability that the ball is red. %conditional prob, useing \sum multiplcation rule
\item [(b)] Suppose we draw a random ball from urn I and transfer it to urn III. Then we choose a ball randomly from urn III. What is the probability that we draw a green ball from urn III.
%multiplication rule over the intersection of probs. P(r & 1 cap r &3)=P(R I)*P(R III | R I) + P(R III \cap G I)
\end{itemize}
\pagebreak





\noindent \fbox{4pts} $~$\textbf{5.} There is a new test for a disease that occurs in about 0.01\% of the population. It detects the disease 99\% of the time. However it has a false positive rate of 10\%. 
\begin{itemize}
\item [(a)] What is the probability that the person's result is positive.		%\sum of conditional prob
\item [(b)] What is the probability that the person actually has the disease if they test positive.		%bayes, use part a 
\end{itemize}
\pagebreak





\noindent \fbox{4pts} $~$\textbf{6.} Suppose $A,B,C\in\mathcal{F},~s.t.~A,B,C$ are mutually independent. \\

Set $P(A)=.1,~P(B)=.2,~P(C)=.3$. 
\begin{itemize}
\item [(a)] $P(A\cap B \cap C^C)$ 		%independence rules
\item [(b)] $P(A\cup C)$			%inclusion/exclusion principle, then independence on P(AC)
\item [(c)] $P((A\cup C)\cap B)$
\item [(d)] Prove that $A\cup C$ is independent of B.
\end{itemize}
\pagebreak



















\end{document}






