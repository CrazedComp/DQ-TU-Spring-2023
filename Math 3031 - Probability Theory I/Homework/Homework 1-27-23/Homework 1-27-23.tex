% --------------------------------------------------------------
% This is all preamble stuff that you don't have to worry about.
% Head down to where it says "Start here"
% --------------------------------------------------------------
 
\documentclass[12pt]{article}
 
\usepackage[margin=1in]{geometry} 
\usepackage{amsmath,amsthm,amssymb}
 
\newcommand{\N}{\mathbb{N}}
\newcommand{\Z}{\mathbb{Z}}
 
\newenvironment{theorem}[2][Theorem]{\begin{trivlist}
\item[\hskip \labelsep {\bfseries #1}\hskip \labelsep {\bfseries #2.}]}{\end{trivlist}}
\newenvironment{lemma}[2][Lemma]{\begin{trivlist}
\item[\hskip \labelsep {\bfseries #1}\hskip \labelsep {\bfseries #2.}]}{\end{trivlist}}
\newenvironment{exercise}[2][Exercise]{\begin{trivlist}
\item[\hskip \labelsep {\bfseries #1}\hskip \labelsep {\bfseries #2.}]}{\end{trivlist}}
\newenvironment{reflection}[2][Reflection]{\begin{trivlist}
\item[\hskip \labelsep {\bfseries #1}\hskip \labelsep {\bfseries #2.}]}{\end{trivlist}}
\newenvironment{proposition}[2][Proposition]{\begin{trivlist}
\item[\hskip \labelsep {\bfseries #1}\hskip \labelsep {\bfseries #2.}]}{\end{trivlist}}
\newenvironment{corollary}[2][Corollary]{\begin{trivlist}
\item[\hskip \labelsep {\bfseries #1}\hskip \labelsep {\bfseries #2.}]}{\end{trivlist}}
 
\begin{document}
 
% --------------------------------------------------------------
%                         Start here
% --------------------------------------------------------------
 
%\renewcommand{\qedsymbol}{\filledbox}
 
\title{Homework 1-24-23}%replace X with the appropriate number
\author{Dean Quach\\ %replace with your name
MATH 2121 - Mathematical Modeling and Simulation} %if necessary, replace with your course title
\date{Janurary 24, 2023}
\maketitle

\hrule
\vspace{20pt}



\textbf{Find examples of ABM's, pick one you want to share with the class on \\Tuesday. You can use the provided resources or the internet for your research.}\\

\noindent For your chosen ABM, note the following:
\begin{enumerate}
\item What are the agents?
\item What are their attributes?
\item What is the environment?
\item What type of interactions are taking place? How often?
\item What questions does the ABM seek to answer?
\item What are the results from the simulation?
\end{enumerate} 

$Sol^n:$ John Conway's "Game of Life"
- It wasn't called an agent based model when it was originally created. 
- But it is the model most people refer to when it comes to the concept of an ABM 
- its like the abstraction/generalization/pure math version of ABM's. which interestingly has many parallel's to video games
- Funfact: it is called a "0-player game", does not require outside input like us holding a nob and tweaking it, you just let it do its thing until you see patterns, the game stops/dies, or chaos/infinity.

---

1. What are the agents?
	- Grid Squares, "you can think of this as groups of people"
2. What are their attributes?
	- Alive/On, or Dead/Off
3. What is the environment?
	- An infinite 2-dimensional checkerboard (of equal spacing)
4. What type of interactions are taking place? How often?
	- Note: each square has 8 squares "next to/touching" it
	- Original "Rules/Interactions" (people have made many other rules, since this is an ABM, once you have a model you can change rules/tweak numbers)
		- Any *living* cell with *fewer than 2* dies. (dies of isolation)
		- Any *living* cell with *2-3 live* neighbors continue to *live*. (lives on to the next generation)
		- Any *living* cell with *more than 3* dies. (dies of overpopulation)
			- Any *dead* cell with *3 live* neighbors becomes alive again. (reproduce)
		- Its time "interval" is more just turned based. 
			- $t=0$:= initial state (with whatever pattern you'd like to start with)
			- $t=1$:= the first changed state
			- $t=2$:= the second state...
			- ...
			- $t=n,~for~some~n\in \mathbb{N}$
			- $t=\infty$ ?
1. What questions does the ABM seek to answer?
	- Stanislaw Ulam was working at the Los Alamos national laboratory (sante fe new mexico) 1940's he was trying to study the growth of crystals, 
	- John von Neumann also was working on self replicating systems and the notion of a robot building another robot. the kinematic model.
		- ended up writting "the general and logical theory of automata"
		- ulam then recomeded using a discrete system
		- Invented first system of cellular automata
			- created an existence proof that a patern would make endless copies of its self.
	- But back to Conway, inspired by this, 
		- he made many two dimensional cellular automaton rules, 
			- interested in unpredicatble cellular automaton
			- wanted to find rules that were simple, alow patterns to emerge, "apparently" grow without limit
			- make it **specifically "difficult to prove"**
1. What are the results from the simulation?
	- a.  describe how life can evolve from an intial state, but more abstractly
	- b.  the answer is, this is its another example of the undecidability of mathematics 
		- Godel's Incompletness theorem
			- Complete X
				- In any system of mathematics with basic arithmetic, there will always be true statements, that are impossible to prove.
			- Consistent?
				- YOU can only have a system that is consistent but not complete
				- SO. Any system of math that is consistent cannot prove its own consistency
			- Decidable X
				- Is there an algorithm that can always a statement always follows from the axioms
					- This whole result is similar to the ABM version of the halting problem developed for the turing machine.
				- Given the initial (even small pattern), you cannot see if a game will do one of the following...
				- "Undecidable", no possible algorithm/method to find what will happen
	- Patterns
		- Some Still lifes 
		- Oscilators
		- Space ships
	- Chaos and grows forever
	- Fizzles out/everyone dies





%\begin{theorem}{2.15} %You can use theorem, exercise, problem, or question here.  Modify x.yz to be whatever number you are proving
%An odd integer times an odd integer results in an odd integer.
%\end{theorem}
% 
%\begin{proof} %You can also use solution in place of proof.
%Assume m and n are both odd integers.\\
%Let m = 2k+1, and n = 2j+1\\
%So mn = 4kj+2k+2j+1\\
%Which factors into 2(2kj+k+j)+1.\\
%
%
%%Note 1: The * tells LaTeX not to number the lines.  If you remove the *, be sure to remove it below, too.
%%Note 2: Inside the align environment, you do not want to use $-signs.  The reason for this is that this is already a math environment. This is why we have to include \text{} around any text inside the align environment.
%By definition 2.9 (2kj+k+j) is an integer.\\
%so by definition 2.10 mn is an odd integer.\\\\
%\end{proof}
%
%\begin{theorem}{2.12} %You can use theorem, exercise, problem, or question here.  Modify x.yz to be whatever number you are proving
%The product of an odd integer and an even integer is odd.\\\\
%\end{theorem}
%\begin{proof} %You can also use solution in place of proof.
%The product of 2 times 3 is 6, which is an even number.
%
%
%\end{proof}








% --------------------------------------------------------------
%     You don't have to mess with anything below this line.
% --------------------------------------------------------------
 
\end{document}
