% --------------------------------------------------------------
% This is all preamble stuff that you don't have to worry about.
% Head down to where it says "Start here"
% --------------------------------------------------------------
 
\documentclass[12pt]{article}
 
\usepackage[margin=1in]{geometry} 
\usepackage{amsmath,amsthm,amssymb}
 
\newcommand{\N}{\mathbb{N}}
\newcommand{\Z}{\mathbb{Z}}
 
\newenvironment{theorem}[2][Theorem]{\begin{trivlist}
\item[\hskip \labelsep {\bfseries #1}\hskip \labelsep {\bfseries #2.}]}{\end{trivlist}}
\newenvironment{lemma}[2][Lemma]{\begin{trivlist}
\item[\hskip \labelsep {\bfseries #1}\hskip \labelsep {\bfseries #2.}]}{\end{trivlist}}
\newenvironment{exercise}[2][Exercise]{\begin{trivlist}
\item[\hskip \labelsep {\bfseries #1}\hskip \labelsep {\bfseries #2.}]}{\end{trivlist}}
\newenvironment{reflection}[2][Reflection]{\begin{trivlist}
\item[\hskip \labelsep {\bfseries #1}\hskip \labelsep {\bfseries #2.}]}{\end{trivlist}}
\newenvironment{proposition}[2][Proposition]{\begin{trivlist}
\item[\hskip \labelsep {\bfseries #1}\hskip \labelsep {\bfseries #2.}]}{\end{trivlist}}
\newenvironment{corollary}[2][Corollary]{\begin{trivlist}
\item[\hskip \labelsep {\bfseries #1}\hskip \labelsep {\bfseries #2.}]}{\end{trivlist}}


%\renewcommand\qedsymbol{$\blacksquare$} 
%\renewcommand\qedsymbol{q.e.d.} 
%\renewcommand\qedsymbol{$\square$} 


\begin{document}
 
% --------------------------------------------------------------
%                         Start here
% --------------------------------------------------------------
 
%\renewcommand{\qedsymbol}{\filledbox}
 
\title{Homework 5}%replace X with the appropriate number
\author{Dean Quach\\ %replace with your name
MATH 3003 - Theory of Numbers} %if necessary, replace with your course title
\date{Februrary 17, 2023}
\maketitle

\hrule
\vspace{20pt}





\begin{exercise}{3.4.1} Use the Euclidean algorithm to find each of the following greatest common divisors.\footnote{I'll keep this format for the rest of this document. Where the first part (a) I put the extra empty valued steps, and then the other parts (b) through \{some letter in the alphabet\} I'll just do quickly.}
\end{exercise}


\noindent \textbf{Part (a) 45,75}\\
$\mathbf{Sol^n:}$
\begin{align*}
gcd(45,75)&\implies\\
&75=45()+\\
&75=45(1)+30\\
&45=30()+\\
&45=30(1)+15\\
&30=15()+\\
&30=\boxed{15}(2)+\mathbf{0}\\
\therefore gcd(45,75)=&15.\qed
\end{align*}


\noindent \textbf{Part (b) 102,222}\\
$\mathbf{Sol^n:}$
\begin{align*}
gcd(102,222)&\implies\\
222&=102(2)+18\\
102&=18(5)+12\\
18&=12(1)+6\\
12&=\boxed{6}(2)+0\\
\therefore gcd(102,222)&=6.\qed
\end{align*}


\noindent \textbf{Part (c) 666,1414}\\
$\mathbf{Sol^n:}$
\begin{align*}
gcd(666,114)&\implies\\
1414&=666(2)+82\\
666&=82(8)+10\\
82&=10(8)+2\\
10&=\boxed{2}(5)+0\\
\therefore gcd(666,1414)&=2.\qed
\end{align*}


\noindent\textbf{Part (d) 20785,44350}\\
$\mathbf{Sol^n:}$
\begin{align*}
gcd(20785,44350)&\implies\\
44350&=20785(2)+2780\\
20785&=2780(7)+1325\\
2780&=1325(2)+130\\
1325&=130(10)+25\\
130&=25(5)+5\\
25&=\boxed{5}(5)+0\\
\therefore gcd(20785,44350)&=5.\qed\\
\end{align*}





\begin{exercise}{3.4.3} For each pair of integers in Exercise 1, express the greatest common divisor of the integers as a linear combination of these integers.
\end{exercise}


\noindent \textbf{Part (a) 45,75.} \\
$\mathbf{Sol^n:}$
\begin{align*}
\intertext{We know}
15&=(1)45+(-1)30
\intertext{and since $30=75-45$,}
15&=45-(75-45)=2(45)-75.
\intertext{$\therefore 15=2(45)-75.\qed$}
\end{align*}


\newpage
\noindent\textbf{Part (b) 102,222}\\
$\mathbf{Sol^n:}$
\begin{align*}
\intertext{We know}
gcd(102,222)=6&=18-12
\intertext{and also $102=18(5)+12\implies12=102-18(5)$. So,}
6&=18-[102-18(5)]=6(18)-102
\intertext{Now $222=102(2)+18\implies 18=222-102(2)$. So, }
6&=6[222-102(2)]-102=6(222)+(-13)(102)
\intertext{$\therefore 6=6(222)-13(102).\qed$}
\end{align*}


\noindent \textbf{Part (c) 666,1414}\\
$\mathbf{Sol^n:}$
\begin{align*}
\intertext{We have $82=10(8)+2\implies$}
2&=82-10(8)
\intertext{And since $666=82(8)+10\implies 10=666-82(8)$, then}
2&=82-[666-82(8)](8)=(-8)666+65(82).
\intertext{And since $1414=666(2)+82\implies 82=1414-666(2),$ then}
2&=(-8)666+65(1414-666(2))=-138(666)+65(1414).
\intertext{$\therefore 2=-138(666)+65(1414).\qed$}
\end{align*}


\newpage
\noindent \textbf{Part (d) 20785,44350}\\
$\mathbf{Sol^n:}$
\begin{align*}
\intertext{So we have}
5&=130-25(5)
\intertext{And since $25=1325-130(10),$ we know}
5&=130-[1325-130(10)](5)\\
&=51(130)-5(1325)
\intertext{And since $130=2780-1325(2)$, we now know }
5&=51[2780-1325(2)]-5(1325)\\
&=51(2780)-107(1325)
\intertext{And since $1325=20785-2780(7),$ we know}
5&=51(2780)-107[20785-2780(7)]\\
&=-107(20785)+800(2780)
\intertext{And since $2780=44350-20785(2),$ we know}
5&=-107(20785)+800[44350-20785(2)]\\
&=-1707(20785)+800(44350)
\intertext{$\therefore 5=-1707(20785)+800(44350).\qed$}
\end{align*}





\begin{exercise}{3.4.5} Find the greatest common divisor of each of the following sets of integers.
\end{exercise}

\noindent \textbf{Part (a) 6,10,15}\\
$\mathbf{Sol^n:}$
\begin{align*}
\intertext{Notice,}
gcd(6,10,15)&=gcd(6,gcd(10,15))\\
\intertext{Quickly,} 
gcd(10,15)&\implies\\
15&=10(1)+5\\
10&=5(2)+0\\
\implies gcd(10,15)&=5\\
\intertext{Now we have}
gcd(6,(5))&\implies\\
6&=5(1)+1\\
5&=1(5)+0\\
\implies gcd(6,5)&=1.\\
\intertext{Therefore,}
gcd(6,10,15)&=1.\qed\\
\end{align*}


\noindent \textbf{Part (b) 70,98,105}\\
$\mathbf{Sol^n:}$
\begin{align*}
\intertext{Notice,}
gcd(70,98,105)&=gcd(gcd(70,105),98)\\
\intertext{Now,}
gcd(70,105)&\implies\\
105&=70(1)+35\\
70&=35(2)+0\\
\implies gcd(70,105)&=35\\
\intertext{So then,}
gcd(gcd(70,105),98)&=gcd(35,98).
\intertext{And} 
gcd(35,98)&\implies\\
98&=35(2)+28\\
35&=28(1)+7\\
28&=7(4)+0\\
\intertext{So} 
gcd(35,98)&=7\\
\intertext{And}
\therefore gcd(70,98,105)&=7.\qed\\
\end{align*}


\noindent \textbf{Part (c) 280,330,405,490}\\
$\mathbf{Sol^n:}$
\begin{align*}
\intertext{Note two calculations...}
gcd(280,330)&\implies\\
330&=280(1)+50\\
280&=50(5)+30\\
50&=30(1)+20\\
30&=20(1)+10\\
20&=10(2)+0\\
\implies gcd(280,330)&=10.\\
\intertext{And also}
gcd(405,490)&\implies\\
490&=405(1)+85\\
405&=85(4)+65\\
85&=65(1)+20\\
65&=20(3)+5\\
20&=5(4)+0\\
\implies gcd(405,490)&=5.\\
\intertext{Now lets consider,}\\
gcd(280,330,405,490)&=gcd(gcd(280,330),gcd(405,490))\\
&=gcd(10,5)\implies\\
&~~~~~~~~~~10=5(2)+0\\
&\implies gcd(10,5)=5.\\
\intertext{Therefore} 
&gcd(280,330,405,490)=5.\qed\\
\end{align*}





\newpage
\begin{exercise}{3.4.7 } Express the greatest common divisor of each set of numbers in Exercise 5 as a linear combination of the numbers in that set.
\end{exercise}


\noindent \textbf{Part (a) 6,10,15}\\
$\mathbf{Sol^n:}$
\begin{align*}
\intertext{Note for $gcd(10,15)=5,$} 
15&=10(1)+5\implies 5=15-10\\
\intertext{Also $6=5(1)+1\implies 1=6-5.$ So}\\
1&=6-5=6-(15-10)=6-15+10.
\intertext{$\therefore 1=6+10-15. \qed$}\\
\end{align*}


\noindent \textbf{Part (b) 70,98,105}\\
$\mathbf{Sol^n:}$
\begin{align*}
\intertext{Note that for}
gcd(70,98,105)&=7\implies\\
7&=35-28\\
\intertext{And since $28=98-35(2)$}
7&=35-(98-35(2))=3(35)-98\\
\intertext{Now notice} 
gcd(105,70) &\implies\\
35&=105-70\\
\intertext{So} 
7&=3(105-70)-98\\
\therefore 7&=3(105)-3(70)-98.\qed\\
\end{align*}


\noindent \textbf{Part (c) 280,330,405,490}\\
$\mathbf{Sol^n:}$
\begin{align*}
\intertext{Note that for $gcd(10,5)=5$,} 
5&=10-5\tag{*}\\
\intertext{Now notice}
gcd(280,330)&=10 \implies\\
10&=30-\boxed{20}	&	\land~ 20&=50-30 \\
\implies10&=30-(50-30)=2(\boxed{30})-50	&	\land~ 30&=280-5(50)\\
\implies10&=2(280-50(5))-50=2(280)-11(\boxed{50})	&	\land~ 50&=330-280\\
\implies 10&=2(280)-11(330-280)=13(280)-11(330)\\
\intertext{And also} 
gcd(405,490)&=5 \implies\\
5&=65-\boxed{20}(3)	&	\land~20&=85-65\\
5&=65-(85-65)(3)=4(\boxed{65})-3(85)	&	\land~65&=405-85(4)\\
5&=4(405-85(4))-3(85)=4(405)-19(\boxed{85})	&	\land~85&=490-405\\
5&=4(405)-19(490-405)=23(405)-19(490)\\
\intertext{Now finally,}
\therefore 5&=10-5\tag{*}\\
&=[13(280)-11(330)]-[23(405)-19(490)]\\
&=13(280)-11(330)-23(405)+19(490)\qed\\
\end{align*}












 
\end{document}
