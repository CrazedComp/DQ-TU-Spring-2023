\documentclass[12pt]{article}
 
\usepackage[margin=1in]{geometry} 
\usepackage{amsmath,amsthm,amssymb}
 
\newcommand{\N}{\mathbb{N}}
\newcommand{\Z}{\mathbb{Z}}
\newcommand{\R}{\mathbb{R}}
\newcommand{\Q}{\mathbb{Q}} 

\newenvironment{theorem}[2][Theorem]{\begin{trivlist}
\item[\hskip \labelsep {\bfseries #1}\hskip \labelsep {\bfseries #2.}]}{\end{trivlist}}
\newenvironment{lemma}[2][Lemma]{\begin{trivlist}
\item[\hskip \labelsep {\bfseries #1}\hskip \labelsep {\bfseries #2.}]}{\end{trivlist}}
\newenvironment{exercise}[2][Exercise]{\begin{trivlist}
\item[\hskip \labelsep {\bfseries #1}\hskip \labelsep {\bfseries #2.}]}{\end{trivlist}}
\newenvironment{reflection}[2][Reflection]{\begin{trivlist}
\item[\hskip \labelsep {\bfseries #1}\hskip \labelsep {\bfseries #2.}]}{\end{trivlist}}
\newenvironment{proposition}[2][Proposition]{\begin{trivlist}
\item[\hskip \labelsep {\bfseries #1}\hskip \labelsep {\bfseries #2.}]}{\end{trivlist}}
\newenvironment{corollary}[2][Corollary]{\begin{trivlist}
\item[\hskip \labelsep {\bfseries #1}\hskip \labelsep {\bfseries #2.}]}{\end{trivlist}}


%\renewcommand\qedsymbol{$\blacksquare$} 
%\renewcommand\qedsymbol{$q.e.d.$} 
%\renewcommand\qedsymbol{$\square$} 
\usepackage{tikz-cd}


\begin{document}
 

 
%\renewcommand{\qedsymbol}{\filledbox}
 
\title{Homework 4}%replace X with the appropriate number
\author{Dean Quach\\ %replace with your name
MATH 3003 - Theory of Numbers} %if necessary, replace with your course title
\date{Februrary 17, 2023}
\maketitle

\hrule
\vspace{5pt}
I hilariously messed up, I normally have my homework done early... but this time I had done the wrong section. I had done these exercises, but in section 3.4 (instead of 3.3). This is why I didn't print it this morning. (Also to save paper is a better excuse :)
\hrule
\vspace{20pt}




\renewcommand\qedsymbol{$\square$} 
\begin{exercise}{3.3.1} Find the greatest common divisor of each of the following pairs of integers.
\end{exercise}
\textbf{Part (a) 15, 35}\\
$Sol^n:$\\
The positive divisors 15 := $A=\{1,3,5,15\}$.\\
The positive divisors 35 := $B=\{1,5,7,35\}$.
\begin{center}
So $gcd(15,35)=max(A\cap B)=5.\qed$
\end{center}

\noindent \textbf{Part (b) 0, 111}\\
$Sol^n:$\\
Notice all numbers are divisors of 0 since $c\mid 0,~\forall c\in \Z$
\begin{center}
So of course, \(gcd(0,111)=|111|=111.\qed\)
\end{center}

\noindent \textbf{Part (c) -12, 18}\\
$Sol^n:$\\
The positive divisors -12 := $A=\{1,2,3,4,6,12\}$.\\
The positive divisors 18 := $B=\{1,2,3,6,9,18\}$.
\begin{center}
So $gcd(-12,18)=max(A\cap B)=6.\qed$
\end{center}

\noindent \textbf{Part (d) 99, 100}\\
$Sol^n:$\\
The positive divisors 99 := $A=\{1,3,9,11,22,99\}$.\\
The positive divisors 100 := $B=\{1,2,4,5,10,20,25,50,100\}$.
\begin{center}
So $gcd(99,100)=max(A\cap B)=1.\qed$
\end{center}

\noindent \textbf{Part (e) 11, 121}\\
$Sol^n:$\\
The positive divisors 11 := $A=\{1,11\}$.\\
The positive divisors 121 := $B=\{1,11,121\}$.
\begin{center}
So $gcd(11,121)=max(A\cap B)=11.\qed$
\end{center}

\noindent \textbf{Part (f) 100, 102}\\
$Sol^n:$\\
The positive divisors 100 := $A=\{1,2,4,5,10,20,25,50,100\}$.\\
The positive divisors 18 := $B=\{1,2,3,6,17,34,51,102\}$.
\begin{center}
So $gcd(100,102)=max(A\cap B)=2.\qed$
\end{center}


\renewcommand\qedsymbol{$q.e.d.$} 
\begin{exercise}{3.3.8} Show that the greatest common divisor of an even number and an odd number is odd.
\end{exercise}
\begin{proof}
Let $a,b\in \Z$ such that $a$ is even and $b$ is odd.\\
Then $a=2n,$ for some $n\in \Z$. Also $b=2m+1$ for some $m\in \Z$.\\
Now consider $gcd(a,b)$.\\\\
BWOC, assume that $gcd(a,b)$ is even, and let $gcd(a,b)=d$.\\
Now we know $2\mid d$. Also by the definition of $gcd$, we know $d \mid a ~\land~ d\mid b$.\\
Notice, by transitivity, $2\mid d ~\land~ d\mid a \implies 2\mid a$.\\
But then, by transitivity, $2\mid d ~\land~ d\mid b \implies 2\mid b$. \\
Contradiction! Since $b$ is odd $\implies 2\nmid b$.\\

\noindent Therefore, $gcd(a,b)$ is odd.
\end{proof}




\begin{exercise}{3.3.9} Show that if $a$ and $b$ are integers, not both 0, and $c$ is a nonzero integer, then $gcd(ca,cb)=|c|gcd(a,b).$
\end{exercise}
\begin{proof}
Let $a,b\in \Z$ such that they cannot both be 0. Also let $c\in \Z \backslash \{0\}$.\\
Then $gcd(ca,cb)=d$, where $d\in\Z_{\geq0}$. \\

\noindent Now by Bezout's Thm., we know
\begin{align*}
d&=n(ca)+m(cb)\text{, for some $n,m\in\Z$}\\
&=c(na)+c(mb)\\
&=c(na+mb)
\end{align*}
Small note, we let $c\in \Z\{0\}$
which includes $\Z_{<0}$. And since $gcd\implies max$, \\even though divisors of $a$ and $b$ are both positive and negative, \\we only need to consider the positive divisors.
Hence, $d=|c|(na+mb)$.\\

\noindent Therefore $gcd(ca,cb)=|c|gcd(a,b)$.
\end{proof}


\newpage
\begin{exercise}{3.3.10} Show that if $a$ and $b$ are integers, with $gcd(a,b)=1,$ then \\$gcd(a+b,a-b)=1 ~\lor~2.$ 
\end{exercise}
\begin{proof}
Let $a,b\in \Z$ such that $gcd(a,b)=1$. Also let $d=gcd(a+b,a-b).$\\
Then $d\mid a+b~\land~a-b$\\
and then $d\mid (a+b)n+(a-b)m,$ for some $n,m\in\Z.$
Observe $d\mid 2$ since $d=gcd(2a,2b)=2gcd(a,b)=2(1)=2$. Also we know $d\mid (a+b)n+(a-b)m$. So one can choose\\
$m=n=1 \implies 2\mid 2a$, and $m=1, n=-1\implies 2\mid 2b$.\\
Therefore $d\mid 2 \implies d=\{1,2\}.$ 
\end{proof}





\renewcommand\qedsymbol{$\square$} 
\begin{exercise}{3.3.17} Find a set of three integers that are mutually relatively prime, but any two of which are not relatively prime. (No ex.'s from text)
\end{exercise}
$Sol^n:$\\
This may already exist, but I had found an interesting way to generate sets of integers $\{a_1,a_2,a_3\}$ with these properties.
Let $p_i$ be a prime number. Then consider the following diagram. 

\begin{center}
\begin{tikzcd}[row sep=large,column sep=huge]
p_i \arrow[d, "1" description, Rightarrow] \arrow[rd, "2" description] & p_{i+1} \arrow[ld, "1" description, Rightarrow, bend left] \arrow[rd, "3" description, dotted, bend right] & p_{i+2} \arrow[ld, "2" description] \arrow[d, "3" description, dotted] \\
a_1                                                                    & a_2                                                                                                        & a_3                                                                   
\end{tikzcd}
\end{center}

I cannot prove this, nor do I want to. But it generates the triple of integers when you multiple the two primes on each arrow type/numbered. For example, the textbook one (which I had accidentally recreated using this method):

\begin{center}
\begin{tikzcd}[row sep=large,column sep=huge]
p_1=3 \arrow[d, "1" description, Rightarrow] \arrow[rd, "2" description] & p_2=5 \arrow[ld, "1" description, Rightarrow, bend left] \arrow[rd, "3" description, dotted, bend right] & p_3=7 \arrow[ld, "2" description] \arrow[d, "3" description, dotted] \\
a_1=15                                                                    & a_2=21                                                                                                        & a_3=35                                                                   
\end{tikzcd}
\end{center}

Which of course has the properties
\begin{itemize}
\item $gcd(15,21,35)=1$
\item $gcd(15,21)=3\neq1$
\item $gcd(15,35)=5\neq1$
\item $gcd(21,35)=7\neq1$
\end{itemize}

I, of course, then checked if this was an example from the text, and indeed was. So I chose a few more primes. (I had also checked for $gcd$ properties by hand, but the question doesn't technically ask for it (and I'm lazy).

\begin{center}
\begin{tikzcd}[row sep=large,column sep=huge]
p_1=13 \arrow[d, "1" description, Rightarrow] \arrow[rd, "2" description] & p_2=17 \arrow[ld, "1" description, Rightarrow, bend left] \arrow[rd, "3" description, dotted, bend right] & p_3=23 \arrow[ld, "2" description] \arrow[d, "3" description, dotted] \\
a_1=221                                                                    & a_2=299                                                                                                        & a_3=391                                                                   
\end{tikzcd}
\end{center}

Which of course has the properties
\begin{itemize}
\item $gcd(221,299,391)=1$
\item $gcd(221,299)=13\neq1$
\item $gcd(221,391)=17\neq1$
\item $gcd(299,391)=23\neq1$
\end{itemize}

So one possible set of three integers that are mutually relatively prime, but any two of which are not relatively prime, are \{221,299,391\}. $\qed$





\renewcommand\qedsymbol{$q.e.d.$} 
\begin{exercise}{3.3.24} Show that if $k$ is a positive integer, then $3k+2$ and $5k+3$ are relatively prime.
\end{exercise}
\begin{proof}
Let $k\in \Z_{>0}$. And set $gcd(3k+2,5k+3)=d$ for some $d\in\Z$.\\
Then $d\mid n(3k+2)+m(5k+3)$, for some $n,m\in\Z$. WNTS the dividend is equal to 1.\\
Well take $n=5,~m=-3$. Then \\
$n(3k+2)+m(5k+3)=5(3k+2)+(-3)(5k+3)=1$\\
Hence, $d\mid 1 \implies d=1$. So $gcd(3k+2,5k+3)=d=1$\\

\noindent Therefore $3k+2$ and $5k+3$ are relatively prime.
\end{proof}





\begin{exercise}{3.3.25} Show that if $8a+3$ and $5a+2$ are relatively prime for all integers $a$.
\end{exercise}
\begin{proof}
\footnote{($gcd(a+cb,b)=gcd(ab)$), Also the gcd being commutative steps (on the right) might be overkill.}
\begin{align*}
gcd(8a+3,5a+2)&=gcd((8a+3)+(-1)(5a+2),5a+2)\\
&=gcd(3a+1,5a+2)=gcd(5a+2,3a+1)\\
&=gcd((5a+2)+(-1)(3a+1),3a+1)\\
&=gcd(2a+1,3a+1)=gcd(3a+1,2a+1)\\
&=gcd(3a+1-(2a+1),2a+1)\\
&=gcd(a,2a+1+(-2)(a))\\
&=gcd(a,1)=1.\\
\intertext{Therefore}
gcd(8a+3,5a+2)&=1.
\end{align*}
\end{proof}


%
%\noindent \textbf{Part (c) 280,330,405,490}\\
%$\mathbf{Sol^n:}$
%\begin{align*}
%\intertext{Note that for $gcd(10,5)=5$,} 
%5&=10-5\tag{*}\\
%\intertext{Now notice}
%gcd(280,330)&=10 \implies\\
%10&=30-\boxed{20}	&	\land~ 20&=50-30 \\
%\implies10&=30-(50-30)=2(\boxed{30})-50	&	\land~ 30&=280-5(50)\\
%\implies10&=2(280-50(5))-50=2(280)-11(\boxed{50})	&	\land~ 50&=330-280\\
%\implies 10&=2(280)-11(330-280)=13(280)-11(330)\\
%\intertext{And also} 
%gcd(405,490)&=5 \implies\\
%5&=65-\boxed{20}(3)	&	\land~20&=85-65\\
%5&=65-(85-65)(3)=4(\boxed{65})-3(85)	&	\land~65&=405-85(4)\\
%5&=4(405-85(4))-3(85)=4(405)-19(\boxed{85})	&	\land~85&=490-405\\
%5&=4(405)-19(490-405)=23(405)-19(490)\\
%\intertext{Now finally,}
%\therefore 5&=10-5\tag{*}\\
%&=[13(280)-11(330)]-[23(405)-19(490)]\\
%&=13(280)-11(330)-23(405)+19(490)\qed\\
%\end{align*}












 
\end{document}
